

\section{The gem5 Simulator}

The gem5 simulator~\cite{Binkert-gem5-2011} is an open source community-supported computer architecture simulator system.
It consists of a simulator core and models for everything from out of order processors, to DRAM, to network devices.
The gem5 project consists of the gem5 simulator\footnote{\url{https://gem5.googlesource.com/public/gem5}}, documentation\footnote{\url{https://www.gem5.org/}}, and common resources\footnote{\url{https://gem5.googlesource.com/public/gem5-resources}} that enable computer architecture research.

The gem5 project is governed by a meritocratic, consensus-based community governance document\footnote{\url{https://www.gem5.org/governance/}} with a goal to provide a tool to further the state of the art in computer architecture.


\begin{itemize}
    \item Give a high-level overview of gem5
    \item What are the main features
    \item What sets it apart from other architecture simulators
    \begin{itemize}
        \item Governance and community-oriented
        \item Used by multiple companies and by many academics
    \end{itemize}
\end{itemize}

\subsubsection{gem5's main features}

ISA support.

CPU models that can be used with (almost) any ISA.

Memory system.

Two different cache models.

``Classic caches'': I would like to come up with a better name than this.
Maybe ``Composable caches''?

Ruby allows for user-defined cache coherence protocols.
We provide the following protocols.

Ruby also works with Garnet detailed network model.

Full system support.



\subsection{The past nine years}

The gem5 simulator has been wildly successful in the past nine years since the initial release of gem5.

\begin{itemize}
    \item Talk about the citations to gem5. Lots of use.
    \item Very active development community
    \item Lots of commits
    \item Lots of committers
\end{itemize}

Show a figure with the number of commits per year.

Show a figure with the number of unique contributors per year.

Unfortunately, this success brought growing pains.
Cite "gem5 Horrors"~\cite{Power-gem5horrors-2015}.

\subsection{gem5 now}

To solve these problems we
\begin{itemize}
    \item Instituted a governance structure
    \item Improved documentation (cite Learning gem5)
    \item Worked to fund raise (CCRI)
    \item Moved development model to modern tools (git, gerrit)
    \item Created a contributing document to help people get started
\end{itemize}

\subsubsection{Changes in gem5}

In addition to these systematic changes there has also been lots of improvements to the codebase.
Section~\ref{sec:changes} contains all of the details.

These include... (a long list).

\subsubsection{Using gem5 in research and education}

This talks about how to use gem5 in your research and your education.

Start with \url{gem5.org}.

\subsection{The future of gem5}

The future of gem5 is bright.

\begin{itemize}
    \item Lots of new stuff coming in a new roadmap that's coming soon!
    \begin{itemize}
        \item Improved python interface
        \item New RAM models (including NVM)
        \item Garnet 3.0
        \item Models for ML accelerators
        \item Known-good configurations
        \item Many more.
    \end{itemize}
    \item Here, we talk about how to contribute to gem5.
    \item We will do a better job recognizing contributions. The gem5 project is only successful with the community is vibrant. We have taken positive steps towards making it easier to contribute (like a governance document and contributing documentation), but we need to do more. We will encourage everyone to contribute.
    \item More educational material coming!
    \item More workshops, tutorials, etc.
\end{itemize}
