\subsection[Learning gem5]{Learning gem5\footnote{By Jason Lowe-Power}}

The gem5 simulator has a steep learning curve.
Most of the time, using gem5 in research means \emph{modifying} the simulator to change or add new models.
Not only do new users have to navigate the 100s of different models, but they also have to understand the core of the simulation framework.
We found that this steep learning curve was one of the biggest impediments to productively using gem5.
There was anecdotal evidence that it would take new users \emph{years} to learn to use gem5 effectively~\cite{Power-gem5horrors-2015}.
Additionally, the only way to learn parts of gem5 was to work with a senior graduate student or to intern at a company and pick up the knowledge ``on the job''.
Many parts of gem5 were not documented except as the source code.

\emph{Learning gem5} reduces the knowledge gap between new users and experienced gem5 developers.
\emph{Learning gem5} takes a bottom up approach to teaching new users the internals of gem5.
There are currently three parts of \emph{Learning gem5}, ``Getting Started'', ``Modifying and Extending'', and ``Modeling Cache Coherence with Ruby''.
Each part walks the reader through a step-by-step coding example starting from the simplest possible design up to a more realistic example.
By explaining the thought process behind each step, the reader gets a similar experience to working alongside an experienced gem5 developer.
\emph{Learning gem5} includes documentation on the gem5 website\footnote{\url{http://www.gem5.org/documentation/learning_gem5/introduction/}} and source code in the gem5 repository for these simple ground-up models.

Looking forward, we will be significantly expanding the areas of the simulator covered by \emph{Learning gem5} and creating a gem5 ``summer school'' initially offered summer of 2020.
This ``summer school'' will mainly be an online class (e.g., Coursera) with all videos available on the gem5 YouTube channel\footnote{\url{https://www.youtube.com/channel/UCCpCGEj_835WYmbB0g96lZw}}, but we hope to have in-person versions of the class as well.
These classes will also be the basis of gem5 Tutorials held with major computer architecture and other related conferences.
