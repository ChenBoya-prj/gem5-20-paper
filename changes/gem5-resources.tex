\subsection[gem5 resources]{gem5 resources\footnote{by Ayaz Akram, Hoa Nguyen, Mahyar Samani}}

The gem5 simulator provides support for simulating many different system configurations. However,
setting up the simulator to run experiments, specifically in full system mode, could take significant
amount of time and become very complicated. To improve the reproducibility of gem5 experiments
many details such as code version should be documented so that created components are identical for
reuse.

gem5 resources consist of components used to conduct certain computer systems architecture
research on known-good configurations using gem5. They include anything from the configuration
files used to build a Linux kernel that works with gem5 to the configuration scripts that
describe the computer system to be simulated. The provided resources have been tested with gem5-20
and their working status and initial statistics along with their creation processes have been documented\footnote{\url{https://www.gem5.org/documentation/benchmark_status/}\footnote{\url{https://gem5.googlesource.com/public/gem5-resources}}}.
They could be used to do research with different system configurations and to save the user substantial amount of time.
Moreover, some of the provided resources could be modified per user requirements such as the working Ubuntu 20.04 disk-image.
One of the most important resources required by any full system experiment with gem5 is the disk-image
which has one of the most time consuming and error prone build procedures. The disk-images provided by
gem5 resources have been created by packer. We also provide gem5 resources to conduct experiments with many
popular benchmark suites like SPEC 2006~\cite{spec06}, SPEC 2017~\cite{spec17}, PARSEC~\cite{parsec},
NAS Parallel Benchmarks (NPB)~\cite{npb}, GAP Benhmark Suite (GAPBS)~\cite{gapbs} and the Linux Kernel.
