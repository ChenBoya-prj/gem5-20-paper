\subsection[gem5 resources]{gem5 resources\footnote{by Ayaz Akram, Bobby R. Bruce Hoa Nguyen, and Mahyar Samani}}

While the gem5 simulator permits the simulation of many different systems on
different benchmarks and tests, gathering and compiling the resources do to so
be a laborious process. To provide
a better user-experience we have began maintaining \emph{gem5
resources}; a set of artifacts
that are not required to build or run gem5, but that may be utilized to carry out
particular simulations. For example, Linux kernels, disk images, popular
benchmark suiites, and commonly used tests binaries. As part of our gem5-20
release, these resources, with source code and build instructions for each, are
gradually being centralized in a common repository\footnote{
\url{https://gem5.googlesource.com/public/gem5-resources}}.

A key goal of this repository is to ensure reproducibility of gem5
experiments. As a case in point, it can often be difficult to assertain what exact
kernel coniguration was used to test booting within a particular architecture
simulation. With gem5 resources, the correct kernel can be cited more easily,
thereby improving reproducibility of experiments and tests.

\subsubsection{Testing gem5-20 with gem5 resources}

Another important aim of creating a common set of gem5 resources is to more
regularly test gem5 on a suite of benchmarks and common Linux booting setups.
As part of gem5-20, we have tested the simulator's effectivness
at running SPEC 2006~\cite{spec06}, SPEC 2017~\cite{spec17},
PARSEC~\cite{parsec}, the NAS Parallel Benchmarks (NPB)~\cite{npb},
and the GAP Benchmark Suite (GAPBS)~\cite{gapbs}. We have also shown gem5-20's
performance at running 5 different LTS Linux Kernal releases on a set of
different CPU and memory configurations. This results from these investigations
can be found on our website~\footnote{
\url{http://www.gem5.org/documentation/benchmark_status}}.

We hope that, moving forward, we can use this information, and the more stricter
guarantees of reproducibility given to us by gem5 resources, to better target
problem areas in gem5 in the project. Also, knowledge of what configurations
work best with gem5, with a shared set of common resources, can provide
the community with a set of ``known good'' gem5 setups in which architecture
may be simulated.

%The gem5 simulator provides support for simulating many different system configurations. However,
%setting up the simulator to run simulations, specifically in full system mode, could take significant
%amount of time and become very complicated. In order to ensure the reproducibility of gem5 experiments
%many details such as code version should be documented so that created components are identical for
%reuse. gem5 resources consist of components used to conduct certain computer systems architecture
%research on known-good configurations using gem5. They include anything from the configuration
%files used to build a Linux kernel that works with gem5 to the configuration scripts that
%describe the computer system to be simulated. The provided resources have been tested with gem5-20
%and their working status and initial statistics along with their creation processes have been documented~\cite{benchmark_status}~\cite{resources-repo}.
%They could be used to do research with different system configurations and to save the user substantial amount of time.
%Moreover, some of the provided resources could be modified per user requirements such as the working Ubuntu 20.04 disk-image.
%One of the most important resources required by any full system experiment with gem5 is the disk-image
%which has one of the most time consuming and error prone build procedures. The disk-images provided by
%gem5 resources have been created by packer. We also provide gem5 resources to conduct experiments with many
%popular benchmark suites like SPEC 2006~\cite{spec06}, SPEC 2017~\cite{spec17}, PARSEC~\cite{parsec},
%NAS Parallel Benchmarks (NPB)~\cite{npb}, GAP Benhmark Suite (GAPBS)~\cite{gapbs} and Linux Kernel.
