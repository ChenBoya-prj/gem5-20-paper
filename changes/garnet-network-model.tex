\subsection[Garnet Network Model]{Garnet Network Model\footnote{By Srikant Bharadwaj and Tushar Krishna}}
\label{sec:garnet}

The interconnection system within gem5 is modeled in various levels of detail and provides extensive
flexibility in terms of modeling modern systems.
The interconnect models are present within the cache-coherent Ruby memory system of gem5.
It provides the ability to create arbitrary topologies including both homogeneous and heterogeneous systems.
There are two major variants of network models available within the Ruby memory system today: simple and Garnet.
The simple network models the routers, links, and the latencies involved with minimal detailing.
This is appropriate for simulations that can sacrifice detailed interconnection network modeling for faster simulation.
The Garnet model adds detailed router microarchitecture with cycle-level buffering, resource-contention and flow control mechanisms~\cite{garnet-2}.
This model is suitable for studies that focus on interconnection units and data flow patterns.

Currently, gem5 implements an upgraded Garnet 2.0 model which provides custom routing algorithms, routers/links that support heterogeneous latencies, and standalone network simulation support.
These features allow detailed studies of on-chip networks as well as support for highly flexible topologies.
Garnet is moving to version 3.0 with the release of HeteroGarnet which is underway.
HeteroGarnet revamps Garnet to support the modern heterogeneous systems such as 2.5D integration systems, MCM based architectures, and futuristic interconnect designs such as optical networks~\cite{kite}.
We are also working to include support for recent work on routerless NoCs~\cite{AlazemiABC18, LinPPC20}.

