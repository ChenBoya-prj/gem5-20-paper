\subsection[Testing in gem5]{Testing in gem5\footnote{by Sean Wilson and Bobby R. Bruce}}
\label{sec:testing}

In order to ensure the quality of gem5, we have continued to improve our
testing. Testing ensures errors during development are caught and rectified
early, prior to release. As testing gives certain guarantees that new
modifications do not unintentially break functionality, a good testing
infrastructure is essential to open-source projects, such as gem5, that are
contributed to regularly, from many different individuals with varying levels
of expertise and familiarity with the project.

Due to gem5's age and size, tests give developers a degree of confidence when
undertaking engineering on seldom touched components. They are thereby crucial
in ensuring the smooth running of gem5 project both presently and into the
future.

\subsubsection{The TestLib Framework}

The majority of gem5 tests are run via our \emph{TestLib} testing framework.
This framework enables us to test compiled instances of gem5 on specific
computer architecture simulations, and verify that they run as intended. The
TestLib framework has been designed for tests to be easily added, thereby
allowing tests for new gem5 configurations to be quickly developed. While
powerful, it's important to note that these are system-level tests, and are
therefore not designed to test individual components.

\subsubsection{Unit Tests}

In order to test at a finer granularity, unit tests have been developed and
incorporated into our testing procedures. Unline our TestLib tests, these unit
tests ensure the functionality of gem5 classes and functions. Thus, if broken,
the source of the bugs can more quickly be determined. While a relatively new
component in our testing infrastructure, we are currently in the process of
expanding these tests with the goal of per-SimObject unit tests, and greater
coverage over-all

\subsubsection{Continious Integration}

Via our Gerrit code-review system, we run tests for each
submitted patch prior to incorporation to the gem5 code base. These tests
compile gem5 against key ISA targets, and run a set tests designed to cover all
major gem5 components. This proceedure suppliments our code review system, thereby
reducing the possibility of new contributions introducing bugs.

In addition to the tests run prior to submission, a larger set of tests are
available for more ``in-depth'' testing. These can take several hours complete
execution, and  test functionality such as
full-system booting. We run these tests regularly to ensure gem5 meets our
standards. While these are currently triggered manually, we hope to
automatically run these tests nightly and automatically send error reports to
the developers.
