\subsection[RISC-V ISA Support]{RISC-V ISA Support}

RISC-V is a new ISA which has quickly gained popularity since its creation in 2010, only one year before the initial gem5 release~\cite{Waterman2011riscv}.
In that time, the number of users RISC-V has grown significantly, especially in the computer architecture research community.
Thus, the addition of RISC-V as a supported ISA for gem5 is one of the main new features in the past nine years.

\subsubsection[General RISC-V ISA Implementation]{General RISC-V ISA Implementation\footnote{By Alec Rokle}}

\textbf{TODO: CITE (\url{https://carrv.github.io/2017/papers/roelke-risc5-carrv2017.pdf}) and (\url{https://carrv.github.io/2018/papers/CARRV_2018_paper_3.pdf}).}

The motivation for implementing the RISC-V ISA into gem5 stemmed from needing a way to explore architectural parameters for RISC-V designs.
At the time of implementation, the only means of simulating RISC-V was using spike (its simplified, single-cycle RTL simulator), QEMU, or full RTL simulation or emulation on FPGA.
Spike and QEMU aren’t detailed enough and RTL simulation is too time consuming for these methods to be feasible for architectural parameter exploration, and FPGA emulation is difficult to retrieve performance information from without modifying both the RTL design and executed software to track and present it.
The gem5 simulator provides an easy means of performing this type of analysis through its detailed hardware models that do not require software modification and allows for variable levels of detail.
By adding RISC-V to gem5, this type of analysis is enabled for the rapidly-growing ISA.

The implementation was done by following the divisions of the instruction set into its base ISA and extensions, beginning with the 32-bit integer base set, RV32I.
It was modeled off of the existing gem5 code for MIPS and Alpha, which are also RISC instruction sets that share many of the same operations as RISC-V.
Including support for 64-bit addresses and data (RV64) and for the multiply (M) extension mainly involved adding the new instructions and changing some parameters to expand register and data path widths.
The next two extensions, atomic (A) and floating point (F and D for single- and double-precision, respectively), were more complicated.
The A extension includes both load-reserved/store-conditional (LR/SC) sequence of instructions for performing complex atomic operations on memory and a set of read-modify-write instructions for performing simple ones.
The former two instructions had analogues in MIPS, but, at the time of implementation, gem5 did not have support for single instructions that both read and wrote memory atomically.
These instructions were implemented as a pair of micro-ops that acted like an LR/SC pair with one of the pair additionally performing the specified operation.
Floating-point instructions required many special cases to ensure correct error handling and reporting, and we were not able to implement one of the five possible rounding modes (round away from zero) RISC-V specifies for inexact calculations due to the fact that C++ does not support it.
Finally, support for the non-standard compressed (C) extension, which adds 16-bit versions of high-usage instructions, was added when it was discovered that this extension was included by default in many RISC-V software toolchains.
Its implementation required the creation of a state machine in the instruction decoder to keep track of whether the current instruction is compressed or not, to increment the PC by the correct amount based on the size of the instruction, and to handle cases where a full-length instruction crosses a 32-bit word boundary.

With this implementation, most RISC-V Linux programs are supported for execution in system-call-emulation mode.  Future work by others would then go on to improve the implementation of atomic instructions, including actual atomic read-modify-write accesses in a single instruction and steps toward support for full-system simulation.

\subsubsection[RISC-V Full System Support]{RISC-V Full System Support\footnote{By Nils Asmussen}}

Haven't heard anything back, yet.

